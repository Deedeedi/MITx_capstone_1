\section{Hypothesis tests}

Two hypotheses ($\Theta_0$ disjoint set from $\Theta_1$):
$\begin{cases}
		H_0: \theta \epsilon \Theta_0\\
		H_1: \theta \epsilon \Theta_1\\
\end{cases}$. Goal is to reject $H_0$ using a test statistic.\\

A test $\psi$ has \textbf{level $\alpha$} if $\alpha_{\psi}(\theta) \leq \alpha, \forall \theta \in \Theta_0.$ and \textbf{asymptotic level $\alpha$} if $\lim _{n \to \infty } P_\theta ( \psi = 1) \leq \alpha$.\\

\textbf{A hypothesis-test} has the form 

\begin{align*}
\psi = \textbf{1} \lbrace  T_n \geq c \rbrace
\end{align*}
for some test statistic $T_n$ and threshold $c \in \mathbb{R}$. Threshold $c$ is usually $q_{\alpha/2}$\\
\textbf{Rejection region:}
\begin{align*}
R_{\psi} = \lbrace T_n > c \rbrace
\end{align*}
\textbf{Symmetric about zero and acceptance Region interval:}
\begin{align*}
\psi = \mathbf{1}\{  |T_n| - c > 0 \} .
\end{align*}

\textbf{Power of the test}:\\
\begin{align*}
%\pi_\psi = \inf_{\theta \in \Theta_1}(1 - \beta_\psi(\theta))
$Power = 1 - P(type II Error) = 1 - \beta_\psi(\theta)$
\end{align*}
\textbf{Two-sided test}:
\begin{align*}
H_1&: \theta \neq \Theta_0\\
\mathbf{1}&(|T_ n| > q_{\alpha /2})
\end{align*}
\textbf{One-sided tests}:
\begin{align*}
H_1&: \theta > \Theta_0\\
\mathbf{1}&(T_ n < -q_{\alpha})
H_1&: \theta < \Theta_0\\
\mathbf{1}&(T_ n > q_{\alpha})
\end{align*}
\textbf{Type1 Error:}\\
Test rejects null hypothesis $(statistic \psi = 1)$ but it is actually true $H_0 = TRUE$ also known as the level of a test.\\
\textbf{Type2 Error:}\\
Test does not reject null hypothesis $\psi = 0$ but alternative hypothesis is true $H_1 = TRUE$\\
\textbf{Example:} Let $X_1, \ldots , X_ n \stackrel{i.i.d.}{\sim } \text {Ber}(p^*)$. Question: is $p^* = 1/2$.\\
$H_0: p^* = 1/2; H_1:p^* \neq 1/2$\\
If asymptotic level $\alpha$ then we need to standardize the estimated parameter $\hat{p} = \overline{X}_ n$ first.\\
\begin{align*}
T_n &= \sqrt{n}\frac{\left| \overline{X}_ n - 0.5\right|}{\sqrt{0.5(1 - 0.5)}}\\
\displaystyle \psi _{n} &= \displaystyle \mathbf{1}\left(T_n>q_{\alpha /2} \right)
\end{align*}
where $q_{\alpha /2}$ denotes the $q_{\alpha /2}$ quantile of a standard Gaussian, and $\alpha$ is determined by the required level of $\psi$. Note the absolute value in $T_n$ for this two sided test.\\
\textbf{Pivot:}\\
Let $T_n$ be a function of the random samples $X_1,\dots ,X_ n,\theta$. Let $g(T_ n)$ be a random variable whose distribution is the same for all $\theta$ . Then, $g$ is called a pivotal quantity or a pivot.\\
\textbf{Example:} let $X$ be a random variable with mean $\mu$ and variance $\sigma^2$ . Let $X_1,\dots ,X_ n$ be iid samples of $X$. Then,
\[\displaystyle  g_ n \triangleq \frac{\overline{X_ n} - \mu }{\sigma }\]
is a pivot with $\theta = \left[\mu ~ ~  \sigma ^2\right]^ T$ being the parameter vector (not the same set of paramaters that we use to define a statistical model).
\subsection{P-Value}
The (asymptotic) p-value of a test $\psi_{\alpha}$ is the smallest (asymptotic) level $\alpha$ at which $\psi_{\alpha}$ rejects $H_0$. It is random since it depends on the sample. It can also interpreted as the probability that the test-statistic $T_n$ is realized given the null hypothesis.\\

If $pvalue \leq \alpha$ , $H_0$ is rejected by $\psi_{\alpha}$ at the (asymptotic) level $\alpha$\\

The smaller the p-value, the more confidently one can reject $H_0$.\\
\textbf{Left-tailed p-values:}
\begin{align*}
pvalue&=\mathbb{P}(X\leq x|H_0)\\
&=\mathbf{P}( Z < T_{n,\theta _0}(\overline{X}_ n)))\\
&=\Phi (T_{n,\theta _0}(\overline{X}_ n))\\
&Z\sim \mathcal{N}(0,1)	
\end{align*}
\textbf{Right-tailed p-values:}
\begin{align*}
	pvalue&=\mathbb{P}(X\geq x|H_0)
\end{align*}
\textbf{Two-sided p-values:}
If asymptotic, create normalized $T_n$ using parameters from $H_0$. Then use $T_n$ to get to probabilities.
\begin{align*}
&pvalue=2 min\{\mathbb{P}(X\leq x|H_0),\mathbb{P}(X\geq x|H_0)\}\\
&\mathbb{P}(\lvert Z\rvert > \lvert T_{n,\theta _0}(\overline{X}_ n)\rvert = 2(1-\Phi(T_n))\\
&Z \sim N(0,1)
\end{align*}
\subsection{Comparisons of two proportions}

Let $X_1,\dots ,X_ n \stackrel{iid}{\sim} Bern(p_x)$ and  $Y_1,\dots ,Y_ n \stackrel{iid}{\sim} Bern(p_y)$ and be $X$ independent of $Y$. $\hat{p}_x= 1/n \sum_{i=1}^{n} X_i$ and $\hat{p}_x= 1/n \sum_{i=1}^{n} Y_i$\\

$H_0: p_x = p_y; H_1: p_x \neq p_y$

To get the asymptotic Variance use multivariate Delta-method. Consider $\hat{p}_x - \hat{p}_y = g(\hat{p}_x,\hat{p}_y); g(x,y)= x -y$, then

$\sqrt(n) (g(\hat{p}_x,\hat{p}_y) - g(p_x-p_y)) \xrightarrow[n \rightarrow \infty]{(d)} N(0,\nabla g(p_x-p_y)^T \Sigma \nabla g(p_x-p_y))$\\ 

$\Rightarrow N(0,p_x(1-px) + p_y(1-py))$

\section{Non-asymptotic Hypothesis tests}

\subsection{Chi squared}
The $\chi _ d^2$ distribution with $d$ degrees of freedom is given by the distribution of $Z_1^2 + Z_2^2 + \cdots + Z_ d^2,$ where $Z_1, \ldots , Z_ d \stackrel{iid}{\sim } \mathcal{N}(0,1)$

If $V \sim \chi^2_k:$\\

$\mathbb{E}= \mathbb{E}[Z_1^2] + \mathbb{E}[Z_2^2] + \ldots + \mathbb{E}[Z_d^2] = d$\\ 

$Var(V) = Var(Z_1^2) + Var(Z_2^2) + \ldots + Var(Z_d^2) = 2d$

\textbf{Cochranes Theorem:}\\
If $X_1, ..., X_n \stackrel{iid}{\sim} N(\mu,\sigma^2)$, then sample mean $\bar{X}_n$ and the sample variance $S_n$ are independent. The sum of squares of $n$ variables follows a chi squared distribution with (n-1) degrees of freedom:
\begin{align*}
\frac{n S_ n}{\sigma ^2} \sim \chi _{n -1}^2
\end{align*}
If formula for unbiased sample variance is used:\\
\begin{align*}
\frac{(n-1) S_ n}{\sigma ^2} \sim \chi _{n -1}^2
\end{align*}
\subsection{Student's T Test}
Non-asymptotic hypothesis test for small samples (works on large samples too), data must be gaussian.\\

\textbf{Student's T distribution} with $d$ degrees of freedom:
$t_d := \frac{Z}{\sqrt{V/n}}$ where $Z \sim \mathcal{N}(0,1)$ and $V \sim \chi^2_k$ are independent.\\

\textbf{Student's T test (one sample + two-sided):}\\

Let $X_1, ..., X_n \stackrel{iid}{\sim} N(\mu,\sigma^2)$ and suppose we want to test $H_0: \mu = \mu_0 = 0$ vs. $H_1: \mu \neq 0$.\\

Test statistic follows Student's T distribution:

\begin{align*}
T_n &= \frac{Z}{\tilde{S}} \\
&= \frac{\bar{X}-\mu}{\frac{\hat{\sigma}}{\sqrt{n}}}\\
&=\frac{\displaystyle \sqrt{n}\frac{\bar{X}_n - \mu_0}{\sigma}}{\displaystyle \sqrt{\frac{\tilde{S}_n}{\sigma^2}}}\\
&\sim \frac{N(0,1)}{\sqrt{\frac{\chi^2_{n-1}}{n-1}}}\\
&\sim t_{n-1}
\end{align*}

Works bc. under $H_0$ the numerator $N(0,1)$ and the denominator $\frac{\tilde{S}_n}{\sigma^2} \sim \frac1{n-1}\chi^2_{n-1}$ are independent by Cochran's Theorem.\\

Student's T test at level $\alpha$:
\begin{align*}
\psi_\alpha = \textbf{1}\{|T_n| > q_{\alpha/2}(t_{n-1})\}
\end{align*}

\textbf{Student's T test (one sample, one-sided):}
\begin{align*}
\psi_\alpha = \textbf{1}\{T_n > q_\alpha(t_{n-1})\}
\end{align*}

\textbf{Student's T test (two samples, two-sided):}

Let $X_1, ..., X_n \stackrel{iid}{\sim} N(\mu_X,\sigma^2_X)$ and $Y_1, ..., Y_n \stackrel{iid}{\sim} N(\mu_Y,\sigma^2_Y)$, suppose we want to test $H_0: \mu_X = \mu_Y$ vs $H_1: \mu_X \neq \mu_Y$.

\begin{align*}
T_{n, m} = \frac{\bar{X}_n - \bar{Y}_m}{\displaystyle \sqrt{\frac{\hat{\sigma^2}_X}n + \frac{\hat{\sigma^2}_Y}m}}
\end{align*}


\textbf{Welch-Satterthwaite formula:}\\

When samples are different sizes we need to finde the Student's T distribution of: $T_{n, m} \sim t_N$\\

Calculate the degrees of freedom for $t_N$ with:
\begin{align*}
N = \frac{\displaystyle \left(\frac{\hat{\sigma^2}_X}n + \frac{\hat{\sigma^2}_Y}m\right)^2}{\displaystyle \frac{\hat{\sigma^2}^2_X}{n^2(n-1)} + \frac{\hat{\sigma^2}^2_Y}{m^2(m-1)}} \geq \min(n, m)
\end{align*}

$N$ should be rounded down.

\subsection{Walds Test}

Squared distance of $\widehat{\theta}_ n^{MLE}$ to true $\theta_0$ using the fisher information $I(\widehat{\theta}_ n^{MLE})$ as metric.

Let $\, X_1, \ldots , X_ n \stackrel{iid}{\sim } \mathbf{P}_{\theta ^*}$ for some true parameter $\theta ^* \in \mathbb {R}^ d$ and the  maximum likelihood estimator $\widehat{\theta }_ n^{MLE}$ for $\theta ^*$.\\

Test $H_0: \displaystyle  \theta ^* = \mathbf{0}$ vs $H_1: \displaystyle  \theta ^* \neq \mathbf{0}$\\

Under $H_0$, the asymptotic normality of the MLE $\widehat{\theta }_ n^{MLE}\,$ implies that:\\

$\left\lVert \sqrt{n}\, \mathcal{I}(\mathbf{0})^{1/2}(\widehat{\theta }_ n^{MLE}- \mathbf{0}) \right\rVert ^2 \xrightarrow [n\to \infty ]{(d)} \chi ^2_ d\,$\\

\textbf{Test statistic:}
\begin{align*}
T_n = &
n(\widehat{\theta}_ n^{MLE} - \theta_0)^\top I(\widehat{\theta}_ n^{MLE}) (\widehat{\theta}_ n^{MLE} - \theta_0)\\
&\xrightarrow [n\to \infty ]{(d)} \chi ^2_ d
\end{align*}

\textbf{Wald test} of level $\alpha$:
\begin{align*}
\psi_\alpha = \mathbf{1}\{T_n > q_\alpha(\chi^2_d)\}
\end{align*}

%\textbf{Wald test with one dimensional model}
%\begin{align*}
%W&=\frac{(\widehat{\theta}_ n^{MLE} - \theta_0)^2}{Var(\widehat{\theta})}\\
%&=nI(\widehat{\theta) (\widehat{\theta }^{\text {MLE}} -\theta _0)^2
%\end{align*}

\subsection{Likelihood Ratio Test}
Parameter space $\Theta \subseteq \mathbb{R}^d$ and $H_0$ is that parameters $\theta_{r+1}$ through $\theta_d$ have values $\theta_c^{r+1}$ through $\theta^c_d$ leaving the other $r$ unspecified. That is:\\
$H_0: (\theta_{r+1}, ..., \theta_d)^T = \theta_{r+1...d} = \theta_0$\\

\textbf{Construct two estimators:}
\begin{align*}
\widehat{\theta}_n^{MLE} = argmax_{\theta \in \Theta}(\ell_n(\theta))\\
\widehat{\theta}_n^c = argmax_{\theta \in \Theta_0}(\ell_n(\theta))
\end{align*}

\textbf{Test statistic:}
\begin{align*}
T_n = 2 ( \ell(X_1,..X_n|\widehat{\theta}_n^{MLE}) - \ell(X_1,..X_n|\widehat{\theta}_n^c)))
\end{align*}
\textbf{Wilk's Theorem:} under $H_0$, if the MLE conditions are satisfied: 
\begin{align*}
T_n& \xrightarrow[n \rightarrow \infty]{(d)} \chi_{d-r}^2\\
\end{align*}
\textbf{Likelihood ratio test} at level $\alpha$:
\begin{align*}
\psi_\alpha = \textbf{1}\{T_n > q_\alpha(\chi^2_{d-r})\}
\end{align*}
\subsection{Implicit Testing}
Todo
\subsection{Goodness of Fit Discrete Distributions}

Let $X_1,...,X_n$ be iid samples from a categorical distribution. Test $H_0: p = p^0$ against $H_1: p \neq p^0$. Example: against the uniform distribution $p^0 = (1/K, \ldots, 1/K)^\top$.\\

\textbf{Test statistic} under $H_0$:
\begin{align*}
T_n = n\sum_{k=1}^K\frac{(\hat{p}_k - p^0_k)^2}{p^0_k} \xrightarrow[n \rightarrow \infty]{(d)} \chi^2_{K-1}
\end{align*}
\textbf{Test at level alpha:}
\begin{align*}
\psi_\alpha = \mathbb{1}\{T_n > q_\alpha(\chi^2_{K-1})\}
\end{align*}
\subsection{Kolmogorov-Smirnov test}
\subsection{Kolmogorov-Lilliefors test}
\subsection{QQ plots}
\textbf{Heavier tails}: below > above the diagonal.\\
\textbf{Lighter tails}: above > below the diagonal.\\
\textbf{Right-skewed}: above > below > above the diagonal.\\
\textbf{Left-skewed}: below > above > below the diagonal.\\
